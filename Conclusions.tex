\subsection{What we have seen}
\begin{frame}
    \frametitle{What we have seen}
    \begin{block}{Taxonomy and building blocks}
        \emph{Atomic Consistency, SWMR, Reliable Brodcast}
    \end{block}
    \begin{block}{Shared Memory Algorithms}
        We have seen some intuition
        about what is needed required for providing atomic consistency
        in an Asynchronous system, and a correct algorithm for \emph{BAMP} systems.
    \end{block}
    \begin{block}{Correctness Proof}
        Each of the algorithm's wanted properies has been shown.
    \end{block}
\end{frame}

\subsection{Further Work}
\begin{frame}
    \frametitle{Sequential Consistency too much?}
    \begin{alertblock}{Runtime Limitations}
        Requiring a system to implement Atomic Consistency is a very strong requirement
        and often comes at a steep runtime cost.
    \end{alertblock}
    \begin{block}{Alternative Models: $AC\subseteq SC\subseteq RC$}
        Is an algorithm for (only) Sequential Consistency possible?\\
        Or better yet - an algorithm for Release Consistency with some sort of
        \emph{'fence'} operation?
    \end{block}
\end{frame}

\begin{frame}
    \frametitle{Exploding Serial Numbers}
    Number of messages sent is unbounded, memory complexity
    is logarithmic with number of messages sent (due to counters).
    \begin{block}{Reset Serial Numbers}
        Is it possible to add a mechanism to reset the serial numbers?
    \end{block}
    \begin{block}{Mallicious Serial Numbers}
        Is it possible for byzantine processes to cause the serial numbers (within correct processes) to explode?\\
        If so, is it possible to prevent this?
    \end{block}
\end{frame}

\subsection{The End}
\begin{frame}
    \begin{center}
        \textbf{Thanks for listening!}
    \end{center}
\end{frame}

\subsection{Appendix}
\begin{frame}
    \frametitle{Appendix 1. Algorithm 3 - No Read Inversion}
    \begin{itemize}
        \item Assume $p_2$ starts reading after $p_1$ finishes reading.
        \item Denote the maximal the index read by $p_1$ with $m$.
        \item Some process $p_k$ must have replied to $p_1$ with $m$ as index.
        \item The time of $p_k$ sending reply to $p_1$ is before $p_1$ finishes read, which is before $p_2$ starts read.
        \item When $p_2$ asks $p_k$ to send it's index, $p_k$ will send $m'\geq m$ since serial numbers are ascending.
        \item The index of value returned by $p_2$ is $m''=max(m', \cdot, ...)\geq m'\geq m$.
        \item So the index of $p_2$'s values is at-least that of $p_1$'s value.
    \end{itemize}
\end{frame}