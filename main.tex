\documentclass{beamer}
% basics
\usepackage{amsfonts}
\usepackage{enumitem}
\usepackage{float}
\usepackage{graphicx}
\usepackage{hyperref} 
\usepackage[labelfont=bf]{caption}

% unique math expressions:  
\usepackage{amsmath}
\DeclareMathOperator*{\andloop}{\wedge}
\DeclareMathOperator*{\pr}{Pr}
\DeclareMathOperator*{\approach}{\longrightarrow}
\DeclareMathOperator*{\eq}{=}

% grey paper
\usepackage{xcolor}
% \pagecolor[rgb]{0.11,0.11,0.11}
% \color{white}

% embedded code sections
\usepackage{listings}
\definecolor{codegreen}{rgb}{0,0.6,0}
\definecolor{codegray}{rgb}{0.5,0.5,0.5}
\definecolor{codepurple}{rgb}{0.58,0,0.82}
\lstdefinestyle{mystyle}{
    commentstyle=\color{codegreen},
    keywordstyle=\color{magenta},
    numberstyle=\tiny\color{codegray},
    stringstyle=\color{codepurple},
    basicstyle=\ttfamily\footnotesize,
    breakatwhitespace=false,         
    breaklines=true,                 
    captionpos=b,                    
    keepspaces=true,                 
    numbers=left,                    
    numbersep=5pt,                  
    showspaces=false,                
    showstringspaces=false,
    showtabs=false,                  
    tabsize=2
}
\usetheme{Copenhagen}
\lstset{style=mystyle}


\begin{document}
\title{Atomic Memory in BAMP systems}
\author{Yosef Goren}
\date{}
\frame{\titlepage
On \textbf{'Atomic Read/Write Memory in Signature-Free Byzantine Asynchronous Message-Passing Systems'}\\
A paper by:\\
Achour Mostefaoui, Matoula Petrolia, Michel Raynal, Claude Jard}

\begin{frame}
    \frametitle{\emph{BAMP}: Byzantine Asynchronous Message Passing}
    \begin{itemize}
        \item There are up to $t<\frac{n}{3}$ \textbf{Byzantine Processes}.
        \item Messages are delivered \textbf{Asynchronously}.
        \item Processes communicate by \textbf{Message Passing} (Clique).
        \item '\emph{Signature Free}' - No cryptographic primitives required.
    \end{itemize}
\end{frame}


\begin{frame}
    \frametitle{Who cares?}
    \begin{block}{What we get}
        This implementation provides a reduction from Message Passing models to Atomic Memory models.
    \end{block}
    \begin{block}{What it can be used for}
        Many distributed algorithms are based on atomic memory; this reduction provides instant implementations
        of these algorithms in message passing systems.
    \end{block}
    \begin{examples}
        \begin{itemize}
            \item - Atomic, multi-writer multi-reader registers 
            \item - Concurrent time-stamp systems
            \item - Variants of 4Lexclusion
            \item - Atomic snapshot scan
        \end{itemize}
    \end{examples}
\end{frame}

\begin{frame}
    \frametitle{Atomic Memory (Registers)}
    ... The type of memory we expect.\\
    \textbf{Atomic Concistency} is also known as Linearizability.
    \begin{block}{A Simple Definition}
        \emph{'for any execution of the system, there is some way of totally ordering
        the reads and writes so that the values returned by the reads are the same
        as if the operations had been performed in that order, with no overlapping.'}\\
    \end{block}
    - 'On Interprocess Communication' (1985), Leslie Lamport.
\end{frame}

\begin{frame}
    \frametitle{Single Writer Multiple Reader Registers (\emph{SWMR})}
    A single process can write; everyone can read.\\
    \begin{block}{Single Writer \& Byzantine Processes}
        If all shared memory can be written by all processes - a single Byzantine process can destroy it.
    \end{block}
    \begin{block}{Memory Matrix}
        $\begin{bmatrix}
            1 & 2 & 3\\
            a & b & c
        \end{bmatrix}$
    \end{block}
\end{frame}
\end{document}