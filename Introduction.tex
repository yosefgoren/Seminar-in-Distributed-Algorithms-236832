\subsection{Background}
\begin{frame}
    \frametitle{Why care about implementing registers?}
    \begin{block}{What we get}
        This implementation provides a reduction from Message Passing models to Atomic Concistency Memory models.
    \end{block}
    \begin{block}{What it can be used for}
        Many distributed algorithms are based on atomic memory; this reduction provides instant implementations
        of these algorithms in message passing systems.
    \end{block}
    \begin{examples}
        \begin{itemize}
            \item - Atomic, multi-writer multi-reader registers 
            \item - Concurrent time-stamp systems
            \item - Atomic snapshot scan
        \end{itemize}
    \end{examples}
\end{frame}

\begin{frame}
    \frametitle{Previous Atomic Register Algorithms}
    %TODO: add ABD, others...
\end{frame}

\subsection{System Model}
\begin{frame}
    \frametitle{\emph{BAMP}: Byzantine Asynchronous Message Passing}
    A distributed system of $n$ processes $p_1, p_2, ... p_n$.
    \begin{block}{Byzantine}
        A byzantine process is one that acts arbitrarily, it may crash or even
        send 'malicious' messages to correct processes.\\
        Let $t$ be the number of byzantine processes, we assume $t<\frac{n}{3}$.
    \end{block}
    \begin{block}{Asynchronous}
        A message sent from $p_i$ to $p_j$ may take any amount of time to arrive.
    \end{block}
\end{frame}

\begin{frame}
    \frametitle{Signature Free}
    Many algorithms cope with byzantine processes by requiring them to sign messages,
    thus requiring assuming cryptographic primitives to be correct, it is not the case here.
\end{frame}

\begin{frame}
    \frametitle{Rliable Broadcast Abstraction}
    %TODO: cite the paper properly.
    Based on a seperate paper, we can use a reliable broadcast algorithm as a 'black box' (in \emph{BAMP} systems).
    \begin{block}{Guarantees}
        The reliable broadcast will have syntax '\emph{r\_brodcast m}', and guarantees
        that the message $m$ arrives at all correct processes eventually.
    \end{block}
\end{frame}

\subsection{Specifications}
\begin{frame}
    \frametitle{Single Writer Multiple Reader Registers (\emph{SWMR})}
    A single process can write; everyone can read.\\
    \begin{block}{Single Writer \& Byzantine Processes}
        If all shared memory can be written by all processes - a single Byzantine process can destroy it.
    \end{block}
    \begin{block}{Local Copies}
    Each process $p_i$ has $Reg_i$, but can only write to \alert{$Reg_i[i]$}.
        \begin{center}
            \begin{tabular}{|c|c|c|}
                \hline
                $p_1$ & $p_2$ & $p_3$ \\
                \hline
                \alert{$Reg_1[1]$} & $Reg_2[1]$ & $Reg_3[1]$ \\  
                $Reg_1[2]$ & \alert{$Reg_2[2]$} & $Reg_3[2]$ \\  
                $Reg_1[3]$ & $Reg_2[3]$ & \alert{$Reg_3[3]$} \\
                \hline
            \end{tabular}\\
        \end{center}
    \end{block}
\end{frame}
\begin{frame}
    \frametitle{Atomic Concistency}
    \textbf{Atomic Concistency} requires no concurrect actions to be interleaved,
    it is also kown as \textbf{Linearizability}.
    \begin{block}{(Almost) Formal Definition}
        \emph{'for any execution of the system, there is some way of totally ordering
        the reads and writes so that the values returned by the reads are the same
        as if the operations had been performed in that order, with no overlapping.'}\\
        - 'On Interprocess Communication', Lamport (1985).
    \end{block}
\end{frame}
\begin{frame}
    \frametitle{Notations}
    We define these notations for any correct processes $p_i, p_j$:
    \begin{block}{Reads}
        $read[i,j,x]$ will refer to an invocation by $p_i$, to read $Reg_i[j]$
        which returns the $x$'th value written by $p_j$.
    \end{block}
    \begin{block}{Writes}
        $write[i,y]$ will refer to the $y$'th invocation by $p_i$, to write $Reg_i[i]$.
    \end{block}

\end{frame}
\begin{frame}
    \frametitle{Termination Requirment}
    Let $p_i$ ne a correct process.
    \begin{block}{Write Termination}
        Each invocation of $Reg_i[i].write()$ terminates.
    \end{block}
    \begin{block}{Read Termination}
        For any $j$, all invocations $Reg_i[j].read()$ terminate.
    \end{block}
\end{frame}
\begin{frame}
    \frametitle{Concistency Requirment}
    Let $p_i, p_j$ be correct processes, and $p_k$ be (possibly) byzantine.
    \begin{block}{Associated sequence}
        We can associate a sequence $H_k[x]$ with $p_k$, s.t. if $p_k$ is correct,
        $H_k[x]$ is the value written by $write[k,x]$.
    \end{block}
    \begin{block}{Read followed by Write}
        if $read[j,i,x]$ terminates before $read[i,j,y]$ then $x\leq y$.
    \end{block}
    \begin{block}{Write followed by Read}
        if $write[j,x]$ terminates before $read[i,j,y]$ then $x\leq y$.
    \end{block}
    \begin{block}{No Read inversion}
        if $read[i,k,x]$ terminates before $read[j,k,y]$ starts then $x\leq y$.
    \end{block}

    
\end{frame}